\chapter{Reflection}

\section{Reification / Reflection}

Reification turns a program structure into data that can be reasoned
or acted upon.

Reflection turns data back into a program structure so that it can
affect the computation.

\section{Reflective Towers of Interpreters}

Reflective towers of interpreters have a long history, starting with
Brian Cantwell Smith's 3-LISP, then Brown, Blond and Black.

Each level (including the user level) has a meta level. So each level
can be reasoned about and acted upon.

You can play with the Black reflective tower here:
\url{http://io.livecode.ch/learn/readevalprintlove/black}.
See more reflective towers here: \url{https://github.com/namin/reflective-towers}.

\section{Metaobject Protocols}

Metaobject protocols provide an object-oriented system with a
landscape of design options rather than focusing on a single point in
the design space.

\section{Reflection vs. Abstraction}

See examples in Java, where reflection breaks basic expectations: \url{https://github.com/namin/metaprogramming/tree/master/lectures/2-reflection-dangers}

\section{Further Reading}

\cite{Wand1998} gives meaning to {\tt FEXPR} and shows that it precludes
{\emph generally} admissible optimizations.

Reflective towers of interpreters started with \cite{smith-lisp}, then continued with Brown~\citep{brown}, Blond~\citep{blond}, Black~\citep{black,black-gpce}, Pink \& Purple~\citep{collapsing-towers}.

A Meta-Object Protocol (MOP~\citep{Kiczales93metaobjectprotocols}) opens
up the design of the language to the user. \cite{bracha-mirrors}
advocates a pluggable form of such reflection. \cite{oeom} shows how
to design and bootstrap a small flexible object system in steps.

Fun systems of interests because of their malleability and
bootstrapping capabilities include \cite{fisher1970control}, which
opens up all control structures up to defining parsing as a problem in
control; Smalltalk, where the MOP in Smalltalk-72~\citep{smalltalk-72} was so flexible that
some of its power was revoked later in Smalltalk-80~\citep{smalltalk-80}.
Systems with
syntax malleability include IMP~\citep{imp},
Lisp-70~\citep{testerlisp70}
and Meta~II~\citep{meta2}. Self-hosting systems to
study include ichbins~\citep{ichbins} and Maru~\citep{maru}.

\cite{sobel} give an early account of monadic reflection. A modern
incarnation can be found in Meta-F*~\citep{meta-fstar}, where
reflection as a principled effect enables sound mixing of proofs by
tactics and by automatic discharge to an SMT solver. This approach is
related to ``proof by reflection'', a common technique in proof
assistants such as Coq~\citep{coq-reflection,coq-reflection2},
Agda~\citep{agda-reflection,agda-stump,agda-reflection-eng} and HOL~\citep{hol-reflection}. \cite{harrison-reflection} surveys proof by reflection, and favors the LCF approach even for proving in the large.

The idea of proof by reflection, changing theorem proving in the theory into evaluation in the metatheory, dates back to FOL~\citep{weyhrauch1980prolegomena}.

\cite{malenfant} surveys reflection in logic, functional and object-oriented programming. \cite{tanter} reviews reflection for open implementation.
\cite{costantini2002meta} surveys meta-reasoning, discussing seminal approaches such as FOL and 3-LISP.

\cite{metaconnaissances} (in French) argues for reflection, or change
of abstraction level, to solve problems.
