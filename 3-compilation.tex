\chapter{Compilation}

\section{Partial Evaluation}

Partial evaluation optimizes a program by specializing to some known
arguments. The known arguments are static, while unknown arguments are
dynamic. Binding-Time Analysis (BTA) decides whether an expression in
the program is static or dynamic by propagation. BTA can be done
offline (in advance) or online (while specializing)

\subsection{Futamura projections}

Notation from \cite{selfpe}.

\noindent
Let $L$ be a meta function from a program to the function it computes. Let $S$ and $T$ be programming languages.

\noindent
Equation for partial evaluator mix:
$$(P)\ L\ p\ [d_1, d_2] = L\ (L \text{ mix } [p, d_1])\ d_2$$

\noindent
Equation for an $S$-interpreter int written in $L$:
$$(I)\ S \text{ pgm } \text{data } = L \text{ int } [\text{pgm}, \text{ data}]$$

\noindent
Equation for an $S$-to-$T$-compiler comp written in $L$:
$$(C)\ S \text{ pgm } \text{ data } = T\ (L \text{ comp } \text{pgm}) \text{ data}$$

\noindent
Futamura projections:
$$(1)\ L \text{ mix } [\text{int}, \text{ pgm}] = \textbf{ target}$$
$$(2)\ L \text{ mix } [\text{mix}, \text{ int}] = \textbf{ compiler}$$
$$(3)\ L \text{ mix } [\text{mix}, \text{ mix}] = \textbf{ compiler generator}$$

The equations are easily verified using the equations for mix $(P$), and for
interpreters $(I)$ and compilers $(C)$ above.

\ \\

\noindent
Verify $(1)$:
$$S \text{ pgm } \text{data } = L \text{ int } [\text{pgm}, \text{ data}] \textit{ by } (I)$$
$$L \text{ int } [\text{pgm}, \text{ data}] = L (L \text{ mix } [\text{int}, \text{pgm}]) \text{ data} \textit{ by } (P)$$
\begin{center}Therefore, $(L \text{ mix } [\text{int}, \text{ pgm}])$ acts as {\bf target}.\end{center}

\ \\

\noindent
Verify $(2)$:
$$L \text{ mix } [\text{int}, \text{ pgm}] = \textbf{ target } \textit{ by } (1)$$
$$L \text{ mix } [\text{int}, \text{pgm}] = L (L \text{ mix } [\text{mix}, \text{ int}]) \text{ pgm } \textit{ by } (P)$$
\begin{center}Therefore, $(L \text{ mix } [\text{mix}, \text{ int}])$ acts as a {\bf compiler}.\end{center}

\ \\

\noindent
Verify $(3)$:
$$L \text{ mix } [\text{mix}, \text{ int}] = \textbf{ compiler } \textit{by } (2)$$
$$L \text{ mix } [\text{mix}, \text{int}] = L (L \text{ mix } [\text{mix}, \text{ mix}]) \text{ int } \textit{by } (P)$$
\begin{center}Therefore, $(L \text{ mix } [\text{mix}, \text{ mix}])$ acts as a {\bf compiler generator}.\end{center}

\section{Multi-Stage Programming}

Multi-stage programming explicitly separates a program, turned into a
program generator, into stages -- ``now'' / static / code generator
stage vs ``later'' / dynamic / generated code stage. This distinction
can be done syntactically, as in MetaOcaml, or driven by types, as in
LMS. In contrast to partial evaluation, the binding times are thus
explicit and manual.

See \url{https://scala-lms.github.io/tutorials/index.html} for an introduction to LMS.

\section{Turning Interpreters into Compilers}

An interpreter can be mechanically turned into a (naive) compiler
using staging by making the program static and only the input to the
program dynamic.

For example, a staged regular expression matcher makes the regular
expression static and the matched string dynamic, generating code
specialized to one regular expression. See
\url{https://github.com/namin/metaprogramming/tree/master/lectures/3-regexp}
as a starting point in LMS-verify~\citep{lms-verify}.

\section{Collapsing Towers of Interpreters}

Collapsing towers of interpreters can be achieved through stage
polymorphism~\citep{collapsing-towers}.

The more dynamic approach relies on a stage-polymorphic VM, where
operations are lifted or not by dynamic dispatched,
based on the dynamic types of the arguments.
See \url{http://popl18.namin.net}.

The more static approach relies on stage polymorphism driven by types
and optimizations in LMS. Any code, even generated code, can be
instantiated for interpretation or compilation. See
\url{https://github.com/namin/lms-black}.

\section{Further Reading}

The book by \cite{pebook} is the bible of Partial Evaluation. The
Futamura projections were first introduced by
\cite{Futamura:1971,Futamura:1999}. The tutorial by \cite{cook2011tutorial} is a
good starting point, building a small partial evaluator in Haskell.

The SQL to C compiler by \cite{query-pearl} is good starting point for
learning about Lightweight Modular Staging (LMS~\citep{lms}) and using staging to
turn an interpreter into a compiler.

\cite{oleg-tutorial} teaches modular multi-stage programming with
MetaOCaml.
