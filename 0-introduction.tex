\chapter{Introduction}

\section{Metaprogramming}

%TODO: have quotes?

Metaprogramming is writing programs that manipulate programs, as
data or processes. For example,
\begin{itemize}
\item an interpreter takes a program as input data and then turns this
  description into a process;
\item a compiler takes a program as input data and then produces an
  equivalent program as output data -- translating the description
  from one language to another.
\end{itemize}

Metaprogramming is an art, even just to keep all the levels straight.
This course gives some recipes for a more principled approach to
metaprogramming, from art to science.

Why is metaprogramming relevant?

\begin{enumerate}
\item Meta-computation is at the heart of computation, theoretically
and practically even in the possibility of a universal machine and the
realization of a computer.

\item A useful way of programming is starting with a domain, and
building a little language to express a problem and a solution in that
domain. Metaprogramming gives the tooling.

\item Thinking principally about metaprogramming will give you new
approaches to programming. For example,
you will have a recipe to mechanically turn an interpreter into a
compiler. You will acquire new principles, new shortcuts or lenses or
tools for thought and implementation if you will.

\end{enumerate}

Beyond traditional meta-programs such as interpreters and compilers,
metaprogramming taken more broadly
encompasses programming programming, and manipulating processes that
manipulate processes. Reflection comes into play. Also, control -- for
example, a high-level logic search strategy controlling a low-level
SMT solver is a form of metaprogramming.
Ultimately, we want to abstract over abstractions. Metaprogramming
gives us a way of thinking about higher-order abstractions and
meta-linguistic abstractions.

In this course, we will cover techniques around interpretation,
reflection, compilation, embedding, synthesis, and exploration.

\begin{comment}
\section{Scala, Quickly}

Scala has implicit conversions. They enable promoting one type to
another, and also enable the ``pimp your library'' pattern.
\begin{lstlisting}[language=Scala]
case class Complex(r: Int, i: Int) { ... }
implicit def fromInt(r: Int) = Complex(r, 0)
\end{lstlisting}

Scala has implicit parameters. They support type classes as well as
automatically providing information based on context.
\begin{lstlisting}[language=Scala]
// type classes, e.g. Numeric[T], Ordering[T]
def foo[T:Numeric](x: T) = {
  val num = implicitly[Numeric[T]]
  num.plus(x, x)
}
// equivalently:
def foo2[T](x: T)(implicit num: Numeric[T]) {
  num.plus(x,x)
}
// for example, we could define Numeric[Complex]
implicit def numComplex: Numeric[Complex] = ...
\end{lstlisting}

Scala has case classes. They make it easy to define algebraic data types (ADTs).
Here are ADTs for the lambda calculus, and an interpreter from {\tt Term}s to {\tt Val}ues.
We can use trait mixin composition to share the {\tt Const} nodes between {\tt Term}s and {\tt Val}ues.
We use {\tt sealed} classes to get help with exhaustive pattern matching.
The {\tt @} syntax enables us to precisely name part of the pattern match.
Here, it is useful because we've been using precise types (not just the ADT union type) in some parts (for example, a closure contains a lambda).
\begin{lstlisting}[language=Scala]
sealed abstract trait Term
sealed abstract trait Val
case class Var(x: String) extends Term
case class Lam(p: Var, b: Term) extends Term
case class App(a: Term, b: Term) extends Term

type Env = Map[Var,Val]
case class Clo(l: Lam, m: Env) extends Val
case class Const(n: Int) extends Term with Val

def ev(e: Term, m: Env): Val = e match {
  case v@Var(x) => m(v)
  case c@Const(n) => c
  case l@Lam(p, b) => Clo(l, m)
  case App(a, b) => ev(a, m) match {
    case Clo(Lam(param, body), clo_env) =>
      val arg = ev(b, m)
      ev(body, clo_env + (param -> arg))
  }
}
\end{lstlisting}

A case class provides a few helpers in the companion object. Here is
how the desugaring works.
\begin{lstlisting}[language=Scala]
case class Foo(a: Int)
// produces companion object
object Foo {
  def apply(a: Int): Foo = Foo(a)
  def unapply(x: Foo): Some[Int] = Some(x.a)
}
\end{lstlisting}

For more, see \url{https://github.com/namin/metaprogramming/blob/master/lectures/0-scala-intro/intro_done.scala}
\end{comment}

\section{Further Reading}

\begin{comment}
We will be using the Scala programming language for assignments. There
are many books that could be useful but none are required. I recommend
\cite{scala-by-example} as well as the book by
\cite{scala-book-by-odersky}.
\end{comment}

TAPL~\citep{tapl}, SICP~\citep{sicp}, PAIP~\citep{paip}, Art of Prolog~\citep{art-prolog},
Building Problem Solvers~\citep{bps},
Test Driven Development by Example~\citep{tdd},
%Expert F\#~\citep{expertfsharp},
%Valentino Braitenberg's Vehicles~\citep{braitenberg1986vehicles},
the Reasoned Schemer~\citep{mk2005,mk} --- these are a few of my favorite books.
